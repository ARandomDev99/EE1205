% \iffalse
\let\negmedspace\undefined
\let\negthickspace\undefined
\documentclass[journal,12pt,twocolumn]{IEEEtran}
\usepackage{cite}
\usepackage{amsmath,amssymb,amsfonts,amsthm}
\usepackage{algorithmic}
\usepackage{graphicx}
\usepackage{textcomp}
\usepackage{xcolor}
\usepackage{txfonts}
\usepackage{listings}
\usepackage{enumitem}
\usepackage{mathtools}
\usepackage{gensymb}
\usepackage{comment}
\usepackage[breaklinks=true]{hyperref}
\usepackage{tkz-euclide}
\usepackage{listings}
\usepackage{gvv}
\def\inputGnumericTable{}
\usepackage[latin1]{inputenc}
\usepackage{color}
\usepackage{array}
\usepackage{longtable}
\usepackage{calc}
\usepackage{multirow}
\usepackage{hhline}
\usepackage{ifthen}
\usepackage{lscape}

\newtheorem{theorem}{Theorem}[section]
\newtheorem{problem}{Problem}
\newtheorem{proposition}{Proposition}[section]
\newtheorem{lemma}{Lemma}[section]
\newtheorem{corollary}[theorem]{Corollary}
\newtheorem{example}{Example}[section]
\newtheorem{definition}[problem]{Definition}
\newcommand{\BEQA}{\begin{eqnarray}}
\newcommand{\EEQA}{\end{eqnarray}}
\newcommand{\define}{\stackrel{\triangle}{=}}
\theoremstyle{remark}
\newtheorem{rem}{Remark}
\begin{document}

\bibliographystyle{IEEEtran}
\vspace{3cm}

\title{GATE 2021 EC 23}
\author{EE23BTECH11007 - Aneesh Kadiyala$^{*}$% <-this % stops a space
}
\maketitle
\newpage
\bigskip

\renewcommand{\thefigure}{\theenumi}
\renewcommand{\thetable}{\theenumi}
%fi

\vspace{3cm}
\textbf{Question:} A speech signal, band limited to 4 kHz, is sampled at 1.25 times the Nyquist rate. The speech samples, assumed to be statistically independent and uniformly distributed in the range -5 V to +5 V, are subsequently quantized in an 8-bit uniform quantizer and then over a voice-grade AWGN telephone channel. If the ratio of transmitted signal power to channel noise power is 26 dB, the minimum channel bandwidth required to ensure reliable transmission of the signal with arbitrarily small probability of transmission error (\textit{rounded off to one decimal place}) is \rule{1cm}{0.15mm} kHz.

\hfill(GATE 2021 EC)
\\
\solution
\\
\begin{table}[h!]
    \centering
    \caption{Input Parameters}
    \label{tab:2021ec23_1}
    \begin{tabular}{ | c | c | c | }
    \hline
    Parameter & Value & Description \\
    \hline
    $x(0)$ & 30 & Initial no. of bacteria\\
    \hline
    $r$ & 2 & Ratio of no. of bacteria at end of \\
    & & hour to start of hour (Common Ratio) \\
    \hline
    $x(n)$ & $r^nx(0)u(n)$ & $n^{th}$ term of the GP \\
    \hline
\end{tabular}
\end{table}

The signal is band limited to 4 kHz.
Let the bandwidth of the signal be $B_0$.
\begin{align}
B_0 &= 4\text{kHz} \\
\text{Nyquist Rate } R_N &= 2B_0 \\
\implies R_N &= 8\text{kHz} \\
\text{Sampling Frequency } f_s &= 1.25R_N \\
\implies f_s &= 10\text{kHz} \\
\text{Data Rate } R &= nf_s \\
&= \brak{8}\brak{10\text{kHz}} \\
\implies R &= \brak{8}\brak{10^4}\text{ bits/second}
\end{align}
Channel capacity for an Additive White Gaussian Noise channel is
\begin{align}
C = B\log_2{\brak{1 + \frac{P}{N}}}\text{ bits/second}
\end{align}
where $P$ is the maximum channel power and $N$ is the noise power and $B$ is the channel bandwidth.
Given, signal power to channel noise power is 26 dB.
\begin{align}
\implies 10\log_{10}{\frac{P}{N}} &= 26\text{dB} \\
\implies \frac{P}{N} &= 10^{2.6} \\
&\approx 398.107
\end{align}
For reliable transmission:
\begin{align}
R &\le C \\
8\brak{10^4} &\le B\log_2{399.107} \\
B &\ge \frac{8\brak{10^4}}{\log_2{399.107}} \\
\implies B &\ge 9258.58\text{Hz}
\end{align}
$\therefore$ the minimum channel bandwidth required to ensure reliable transmission of the signal is $\approx9.26$ kHz.
\end{document}