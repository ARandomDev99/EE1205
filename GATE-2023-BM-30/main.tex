% \iffalse
\let\negmedspace\undefined
\let\negthickspace\undefined
\documentclass[journal,12pt,twocolumn]{IEEEtran}
\usepackage{cite}
\usepackage{amsmath,amssymb,amsfonts,amsthm}
\usepackage{algorithmic}
\usepackage{graphicx}
\usepackage{textcomp}
\usepackage{xcolor}
\usepackage{txfonts}
\usepackage{listings}
\usepackage{enumitem}
\usepackage{mathtools}
\usepackage{gensymb}
\usepackage{comment}
\usepackage[breaklinks=true]{hyperref}
\usepackage{tkz-euclide}
\usepackage{listings}
\usepackage{gvv}
\def\inputGnumericTable{}
\usepackage[latin1]{inputenc}
\usepackage{color}
\usepackage{array}
\usepackage{longtable}
\usepackage{calc}
\usepackage{multirow}
\usepackage{hhline}
\usepackage{ifthen}
\usepackage{lscape}

\newtheorem{theorem}{Theorem}[section]
\newtheorem{problem}{Problem}
\newtheorem{proposition}{Proposition}[section]
\newtheorem{lemma}{Lemma}[section]
\newtheorem{corollary}[theorem]{Corollary}
\newtheorem{example}{Example}[section]
\newtheorem{definition}[problem]{Definition}
\newcommand{\BEQA}{\begin{eqnarray}}
\newcommand{\EEQA}{\end{eqnarray}}
\newcommand{\define}{\stackrel{\triangle}{=}}
\theoremstyle{remark}
\newtheorem{rem}{Remark}
\begin{document}

\bibliographystyle{IEEEtran}
\vspace{3cm}

\title{GATE 2023 BM 30}
\author{EE23BTECH11007 - Aneesh Kadiyala$^{*}$% <-this % stops a space
}
\maketitle
\newpage
\bigskip

\renewcommand{\thefigure}{\theenumi}
\renewcommand{\thetable}{\theenumi}
%fi

\vspace{3cm}
\textbf{Question:} In the following circuit, the switch S is open for $t < 0$ and closed for $t \ge 0$.
What is the steady state voltage (in Volts) across the capacitor when the switch is closed?
\begin{figure}[h!]
    \centering
    \includegraphics[width = \columnwidth]{figs/c_fig1.pdf}
\end{figure}
\\
\solution
\\
In steady state, no current flows through capacitor.
\begin{align}
i_2 &= 0 \\
\implies i &= i_1 \\
v_c &= \brak{7\text{k}\ohm}\brak{i_1} \\
10\text{V} &= \brak{10\text{k}\ohm}{i_1} \\
\implies i_1 &= 1\text{mA} \\
v_c &= 7\text{V}
\end{align}
In s-domain:
\begin{figure}[h!]
    \centering
    \includegraphics[width=\columnwidth]{figs/c_fig2.pdf}
\end{figure}
\begin{align}
(7\text{k}\ohm)i_1 &= 10\text{V} - (3\text{k}\ohm)i \\
(10\text{k}\ohm)i_1 &= 10\text{V} - (3\text{k}\ohm)i_2 \\
\implies i_1 &= \frac{10\text{V} - (3\text{k}\ohm)i_2}{10\text{k}\ohm} \\
&= 1\text{mA} - 0.3i_2\\
(7\text{k}\ohm)i_1 &= (10\text{k}\ohm)i_2 + v_c \\
7\text{V} - (2.1\text{k}\ohm)i_2 &= (10\text{k}\ohm)i_2 + v_c \\
\implies v_c + 12100i_2 &= 7
\end{align}
where $i_2 = C\frac{dv_c}{dt}$.
\begin{align}
\implies v_c + 12100C\frac{dv_c}{dt} &= 7 \\
v_c + 0.121\frac{dv_c}{dt} &= 7 \\
\frac{dv_c}{dt} + \frac{v_c}{0.121} &= \frac{7}{0.121} \\
v_ce^{\frac{t}{0.121}} &= \frac{7}{0.121}\int{e^{\frac{t}{0.121}}dt} \\
v_ce^{\frac{t}{0.121}} &= \frac{7}{0.121}\brak{0.121e^{\frac{t}{0.121}} + c}
\end{align}
where $c$ is the integration constant.
Since $v_c = 0$, $c = -0.121$.
\begin{align}
v_c &= 7(1 - e^{-\frac{t}{0.121}})
\end{align}
\end{document}