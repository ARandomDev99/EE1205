% \iffalse
\let\negmedspace\undefined
\let\negthickspace\undefined
\documentclass[journal,12pt,twocolumn]{IEEEtran}
\usepackage{cite}
\usepackage{amsmath,amssymb,amsfonts,amsthm}
\usepackage{algorithmic}
\usepackage{graphicx}
\usepackage{textcomp}
\usepackage{xcolor}
\usepackage{txfonts}
\usepackage{listings}
\usepackage{enumitem}
\usepackage{mathtools}
\usepackage{gensymb}
\usepackage{comment}
\usepackage[breaklinks=true]{hyperref}
\usepackage{tkz-euclide}
\usepackage{listings}
\usepackage{gvv}
\def\inputGnumericTable{}
\usepackage[latin1]{inputenc}
\usepackage{color}
\usepackage{array}
\usepackage{longtable}
\usepackage{calc}
\usepackage{multirow}
\usepackage{hhline}
\usepackage{ifthen}
\usepackage{lscape}

\newtheorem{theorem}{Theorem}[section]
\newtheorem{problem}{Problem}
\newtheorem{proposition}{Proposition}[section]
\newtheorem{lemma}{Lemma}[section]
\newtheorem{corollary}[theorem]{Corollary}
\newtheorem{example}{Example}[section]
\newtheorem{definition}[problem]{Definition}
\newcommand{\BEQA}{\begin{eqnarray}}
\newcommand{\EEQA}{\end{eqnarray}}
\newcommand{\define}{\stackrel{\triangle}{=}}
\theoremstyle{remark}
\newtheorem{rem}{Remark}
\begin{document}

\bibliographystyle{IEEEtran}
\vspace{3cm}

\title{NCERT Discrete Assignment}
\author{EE23BTECH11007 - Aneesh Kadiyala$^{*}$% <-this % stops a space
}
\maketitle
\newpage
\bigskip

\renewcommand{\thefigure}{\theenumi}
\renewcommand{\thetable}{\theenumi}
%fi

\vspace{3cm}
\textbf{Question 11.14.17:} A simple pendulum of length $l$ and having a bob of mass $M$ is suspended in a car. The car is moving on a circular track of radius $R$ with a uniform speed $v$. If the pendulum makes small oscillations in a radial direction about its equilibrium position, what will be its time period?
\\
\solution
\\
The car experiences a centripetal acceleration $a_{centripetal}$ towards the center of the circular track such that
\[a_{centripetal} = \frac{v^2}{R}\]
Inside the car, the bob experiences a pseudo-force called centrifugal force $F_{centrifugal}$ such that
\[F_{centrifugal} = Ma_{centripetal}\]
\[F_{centrifugal} = \frac{Mv^2}{R}\]
\[\implies a_{centrifugal} = \frac{v^2}{R}\]
Time period of a simple pendulum $T$ is given by:
\[T = 2\pi\sqrt{\frac{l}{g_{effective}}}\]
\[T = 2\pi\sqrt{\frac{l}{\sqrt{g^2 + a_{centrifugal}^2}}}\]
\[T = 2\pi\sqrt{\frac{lR}{\sqrt{g^2R^2 + v^4}}}\]
Therefore, the time period of the pendulum is $2\pi\sqrt{\frac{lR}{\sqrt{g^2R^2 + v^4}}}$ seconds.
\end{document}